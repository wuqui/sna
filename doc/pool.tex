\section{Diffusion patterns}

  overall degree of diffusion (synchronic): successful vs. unsuccessful: usage frequency, degree centralization
    no success
    limited
    advanced

  temporal dynamics of diffusion (diachronic)
    stability: stable vs. topical coefficient of variation
    trend
    increasing: potential diffusion
    decreasing: potential centralization
    time window: speed and lifespan


\section{Selection of case study lexemes}

  criteria
    covering clusters of neologism candidates
    frequency counts comparable

  selection
    no diffusion: \emph{microflat}
    limited
    topical: \emph{poppygate}
    centralized: \emph{alt-left}
    decreasing: \emph{solopreneur}
    advanced diffusion:
    advanced: \emph{upcycling}
    increasing: \emph{hyperlocal}

\section{`Interactional' networks}

      The resulting networks are interactional rather than static.~\parencite{Goel2016SocialDynamics} This makes them more similar to communities of practice than to traditional sociolinguistic networks based on static speaker characteristics such socio-economic status.
      In the case of lexical innovation networks that are based on whether speakers provide valuable information. In cases such as \emph{alt-left}, for example, interactional networks show whether usage of the term remains centralized to a tight-knit community of speakers or whether it diffuses to be used by other sub-communities.
      Whether communities are distinct depends on whether users communicate with each other. While the reasons for theses communicative affiliations remain unknown (age, gender, socio-economic status), they are certainly real in mutually engage in communicative interaction. (community = communication)
      It would be interesting to complement this information with static information (e.g. census data \parencite{Eisenstein2014DiffusionLexical}), however such data are currently not obtainable (geotags no longer provided, hard to infer; difficult to predict plus circular (e.g. gender)).
`

\section{Limitations of usage frequency}

  In a strict sense, usage frequency only captures how many tokens of a word were produced by all speakers who have contributed to the corpus at hand. Investigating the degree to which new words diffuse to new speakers and speaker communities on the basis of frequency counts thus depends on several inferences that are commonly accepted as sufficiently reliable.

  \begin{enumerate}
    \item Frequency counts indicate how many speakers have used the term.
    \item The number of speakers who have used the term indicates how many speakers are familiar with the term, whether they have actively used it or not.
    \item The number of speakers who are familiar with the term indicate how many communities of speakers are familiar with the term.
  \end{enumerate}

  These assumptions are to a large extent plausible and have empirically been proven to be effective for investigating both degrees of entrenchment of lexemes in individual speakers as well their conventionality in the speech community.
